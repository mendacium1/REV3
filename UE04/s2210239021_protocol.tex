\documentclass{article}
\usepackage{geometry}
\geometry{
	a4paper,
	%total={170mm,257mm},
	left=20mm,
	top=30mm,
}

\usepackage{fancyhdr}
\usepackage{tikz}
\usepackage{hyperref}
\usepackage{graphicx}
\usepackage{hyperref}
\usepackage{mdframed}
\usepackage{listings} % Include the listings package
\usepackage{xcolor}   % to define your own colors
\usepackage{subcaption}
\usepackage{array}

\bibliographystyle{unsrt}
\bibliography{references}


\newmdenv[
linecolor=blue, % Color of the border line
backgroundcolor=gray!20, % Background color; "gray!20" means "20% gray"
frametitle=Note, % Title of the frame, delete this line if you don't want a title
skipabove=\baselineskip, % Space above the frame
skipbelow=\baselineskip, % Space below the frame
]{mynote}

% code-snippets:
% Define the color styles you wish to use in the document for the Python syntax highlighting
\lstdefinestyle{mystyle}{
	backgroundcolor=\color{white},   % choose the background color; you must add \usepackage{color} or \usepackage{xcolor}
	commentstyle=\color{green},
	keywordstyle=\color{blue},
	numberstyle=\tiny\color{gray},
	stringstyle=\color{red},
	basicstyle=\ttfamily\footnotesize,
	breakatwhitespace=false,         
	breaklines=true,                 
	captionpos=b,                    
	keepspaces=true,                 
	numbers=left,                    
	numbersep=5pt,                  
	showspaces=false,                
	showstringspaces=false,
	showtabs=false,                  
	tabsize=2
}
\definecolor{LightGray}{gray}{0.9}

\lstset{style=mystyle} % Apply your style globally to the document


\newcommand{\LVA}{Reverse Engineering}
\newcommand{\LVAKURZ}{REV3}
\newcommand{\SEMESTER}{WS 2023/2024}
\newcommand{\UELABEL}{UE 04}
\newcommand{\UETITLE}{Dynamische Analyse}
\newcommand{\AUTHOR}{Jakob Mayr}


\title{\vspace{5cm} \LVA\ (\LVAKURZ)\\ \vspace{1cm} \textbf{\UELABEL\ -- \UETITLE\ -- Protokoll} \vspace{2.5cm}}
\author{\AUTHOR}
\date{\SEMESTER}

\begin{document}
	
	\pagestyle{fancy}
	
	\maketitle
	
	\tikz [remember picture, overlay] %
	\node [shift={(3.7cm,-4cm)}] at (current page.north west) %
	[anchor=north west] %
	{\includegraphics{fhooe_logo.jpg}};
	
	\tikz [remember picture, overlay] %
	\node [shift={(10cm,-4.8cm)}] at (current page.north west) %
	[anchor=north west] %
	{\includegraphics{si_logo.jpg}};
	
	%\tikz [remember picture, overlay] %
	%\node [shift={(7.2cm,-11.65cm)}] at (current page.north west) %
	%[anchor=north west] %
	%{\includegraphics[scale=0.12]{./img/star_wars_logo_no_background.png}};
	%
	%\pagebreak
	
	\fancyhf{}
	\fancyhead[L]{\LVA\ (\LVAKURZ)}
	\fancyhead[C]{\UELABEL}
	\fancyhead[R]{\SEMESTER}
	\fancyfoot[L]{Seite \thepage\ von \pageref{LastPage}}
	\fancyfoot[R]{\AUTHOR}
	
	
	\pagebreak
	
	\section*{Aufgabe 1 - Dynamische Analyse Windows}
	\begin{mynote}
		Die Analyse wurde in einer Windows 11 VM unter qemu durchgeführt.
	\end{mynote}
	\subsection*{Process Tree}
	Prozessbäume geben immer einen guten ersten Überblick, was alles passiert (ist):\\
	\includegraphics[width=0.7\linewidth]{"pictures/1.4 process tree"}
	\subsection*{Filtern auf Prozess und Kind-Prozesse}
	Anschließend kann auf einen Prozess und dessen Kind-Prozesse gefiltert werden:\\
	\includegraphics[width=0.5\linewidth]{"pictures/1.5 process tree2"}\\
	Folglich der Filter:\\
	\includegraphics[width=0.5\linewidth]{"pictures/1.6 process tree3"}
	
	\pagebreak
	
	\subsection*{Filtern auf Event-Klassen}
	In ProcMon kann auf Event-Klassen gefilter werden:\\
	\includegraphics[width=0.4\linewidth]{"pictures/1.7 event class"}\\
	
	\pagebreak
	
	\noindent Folgende Event-Klassen mit den jeweiligen Event-Typen konnten auf dem System (allgemein) gefunden werden:\\
	\begin{table}[h]
		\centering
		\begin{tabular}{|m{3.3cm}|m{2.2cm}|m{2.6cm}|m{2.4cm}|m{2.4cm}|}
			\hline
			\textbf{File System} & \textbf{Registry} & \textbf{Network} & \textbf{Process} & \textbf{Thread} \\ \hline
			CreateFile           & QueryValue        & TCP Connect      & Process Create   & Thread Create   \\ \hline
			CloseFile            & SetValue          & UDP Receive      & Process Exit     & Thread Exit     \\ \hline
			ReadFile             & CreateKey         & TCP Disconnect   & Load Image       &                 \\ \hline
			WriteFile            & EnumValue         & UDP Send         & Unload Image     &                 \\ \hline
			DeleteFile           & QueryKey          &                  &                  &                 \\ \hline
			QueryInformationFile &                   &                  &                  &                 \\ \hline
			SetInformationFile   &                   &                  &                  &                 \\ \hline
			RenameFile           &                   &                  &                  &                 \\ \hline
			QueryDirectory       &                   &                  &                  &                 \\ \hline
			QueryEAFile          &                   &                  &                  &                 \\ \hline
			SetEAFile            &                   &                  &                  &                 \\ \hline
			QuerySecurityFile    &                   &                  &                  &                 \\ \hline
			SetSecurityFile      &                   &                  &                  &                 \\ \hline
			CreateFileMapping    &                   &                  &                  &                 \\ \hline
		\end{tabular}
		\caption{ProcMon Event Classes and Types}
		\label{table:procmon-events}
	\end{table}

	\noindent Filter für Event-Klasse "File System":\\
	\includegraphics[width=0.7\linewidth]{"pictures/1.7 event class file"}\\
	Filter für Event-Klasse "Registry":\\
	\includegraphics[width=0.7\linewidth]{"pictures/1.7 event class registry"}\\
	Filter für Event-Klasse "Network":\\
	\includegraphics[width=0.7\linewidth]{"pictures/1.7 event class network"}\\
	
	

	
	\pagebreak
	
	\subsection*{Interpretation der Egebnisse der erzeugten Windows-Executables}
	\textbf{Dateizugriffe}:\\
	\begin{enumerate}
		\item dll's:\\
		Auffällig ist, dass die mit "md" und "mdd" kompilierten Files mehr Datei-Operationen durchgeführt haben. Die Operationen betreffen bei beiden die "vcruntime140.dll" und die "ucrtbased.dll" bei "mdd".
		\item Prefetch:\\
		Alle Varianten bis auf "mtd" beinhalten Datei-Operationen mit dem Pfad "C:\textbackslash Windows\textbackslash Prefetch\textbackslash ...". Dieser Pfad beschleunigt unter Windows 11 das ausführen von Dateien durch vorgeladenene Inhalte, welche in diesem Verzeichnis liegen.
		\item Conhost.exe:\\
		Die Dateizugriffe vom Kind-Prozess "Conhost.exe" sind bei allen Executables gleich.
	\end{enumerate}
	\begin{mynote}
		Allgemein anzumerken ist, dass unter Windows 7 mehr Events zu sehen sind, da das Laden von dlls anders funktioniert. Die \textbf{api-ms-win-*.dll}s tauchen unter Windows 11 nicht auf. Die genauen Gründe für die Unterschiede zwischen Win7 und Win10 konnten nicht recherchiert werden.\\
	\end{mynote}
	
	\noindent Windows 7 Beispiel-Screenshot (md-Variante):\\
	\includegraphics[width=0.7\linewidth]{"pictures/1.7 event class file2"}\\
		
	\noindent Windows 11 Beispiel-Screenshot (md-Variante):\\
	\includegraphics[width=1\linewidth]{"pictures/1.7 event class file3"}\\
	
	\noindent\textbf{Registryzugriffe}\\
	\begin{enumerate}
		\item Operationstypen:\\
		Die einzigen Operationen in allen 4 Varianten sind "RegOpenKey", "RegQueryValue" und "RegCloseKey".
		\item Pfade:\\
		Die Pfade sind großteils in allen 4 Varianten gleich. Die einzigen Unterschiede sind die Anzahl der Events und der Pfad "\texttt{HKLM\textbackslash System\textbackslash CurrentControlSet\textbackslash Control\textbackslash Session Manager\textbackslash SafeDllSearchMode}" welcher in den "mt"- und "mtd"-Varianten nicht vorkommt.\\
		\includegraphics[width=0.9\linewidth]{"pictures/1.7 event class registry2"}
		\item Conhost.exe:\\
		Die Registryzugriffe vom Kind-Prozess "Conhost.exe" sind bei allen Executables gleich.
	\end{enumerate}
	\noindent\textbf{Netzwerkkommunikation}\\
	Filtert man in \textbf{Process Monitor} auf die Event-Klasse "Network" so findet man in allen 4 Varianten keine Ergebnisse. Die Exectuables erstellen keine Sockets und somit gibt es auch keine Netzwerkkommunikation.
	
	\begin{mynote}
		Die .pml-Dateien (Process Monitor) der 4 Varianten sind im zip-Archiv unterschiedlich gefiltert hinterlegt.
	\end{mynote}
	\pagebreak
	
	\section*{Aufgabe 2 - Dynamische Analyse Windows (File: "a")}
	\begin{mynote}
		Die Datei wurde in der zur Verfügung gestellten "Malware VM" analysiert.
	\end{mynote}
	Zu Beginn ist es wieder hilfreich über den Prozessbaum auf den eigentlich zu analysierenden Prozess wie auch dessen Kind-Prozesse zu filtern:\\
	\includegraphics[width=1\linewidth]{"pictures/2.1 main"}
	Anschließend kann es interessant sein, sich eine Summary der Prozessaktivitäten anzeigen zu lassen:\\
	\includegraphics[width=1\linewidth]{"pictures/2.1 main2"}
	Dies verschafft bereits einen groben Überblick. Auffällig ist ebenfalls, dass es keine "Network Events" gibt, obwohl firefox ein YouTube-Video lädt. Dies liegt vermutlich an den Funktionsumfängen des "Process Monitor"s unter Windows 7. Deshalb wird für die Netzwerkanalyse später der "Microsoft Network Monitor 3.4" verwendet.\\
	
	\pagebreak
	
	\noindent Bei der Betrachtung der Prozesserstellungen ist beispielsweise auffällig, wie die "firefox.exe" aufgerufen wird:\\
	\includegraphics[width=1\linewidth]{"pictures/2.1 process creation"}\\
	Zugehörige CommandLine:\\
	\includegraphics[width=1\linewidth]{"pictures/2.1 firefox start"}\\
	Das Ausführen dieses Befehls startet "firefox.exe" (Browser) mit einer übergebenen url. Die Auswirkung des Parameters "-osint" wurde nicht näher recherchiert.
	
	\pagebreak
	
	\noindent\textbf{Dateizugriffe}\\
	Da nach dem Filtern auf "\texttt{Process Name is a.exe}" und "\texttt{Event Class is File System}" immernoch weit über 1000 Events bleiben, kann das Tool "File Summary" im "Process Monitor" hilfreich sein:\\
	\includegraphics[width=1\linewidth]{"pictures/2.2 File Summary"}
	Neben den "normal" interessanten Informationen, wie importierte Funktionen/geladenen Libraries, findet man über die File Summary sehr schnell, dass die firefox.exe verwendet wird und dass mit einem File "sib.html" gearbeitet wird. Anzumerken ist, dass hier nur das File "a" gezeigt wird (mit Kind-Prozessen mehr Information).\\
	Durch einen Doppelclick in der "File Summary" kann auf die entprechenden Events gefiltert werden:\\
	\includegraphics[width=1\linewidth]{"pictures/2.2 File Summary2"}
	Betrachtet man zusätzlich die .htm-Datei im "C:\textbackslash Users\textbackslash user\textbackslash AppData\textbackslash Local\textbackslash Microsoft\textbackslash Windows\textbackslash Temporary Internet Files\textbackslash Content.IE5\textbackslash ..."-Verzeichnis, kann man festellen, dass immer 8.192 Bytes dem File angehängt werden. Dies könnte darauf hindeuten, dass aus dem Internet Informationen stückchenweise heruntergeladen und in die temporäre Datei "sichere-informationssysteme[1].htm" gespeichert werden.\\
	\includegraphics[width=1\linewidth]{"pictures/2.2 File Summary3"}
	
	
	\pagebreak
	
	\noindent \textbf{Registryzugriffe}\\
	Da das "a"-File ebenfalls sehr viele Registryzugriffe hat, ist es am sinnvollsten die "Registry Summary" zu verwenden.\\
	\includegraphics[width=1\linewidth]{"pictures/2.3 Registry Summary"}\\
	Auf Grund der großen Menge an Events und einer fehlenden Expertise, konnte bei den Registryzugriffen keine besonderen Auffälligkeiten gefunden werden.\\
	
	\pagebreak
	
	\noindent \textbf{Netzwerkkommunikation}\\
	Da die Analyse der Netzwerkkommunikation eines Prozesses unter Windows 7 mit dem "Process Monitor" nicht so gut funktioniert wie unter Windows 11, wurde die zustätzliche Software "\textbf{Microsoft Network Monitor 3.4}" dafür verwendet:\\
	\url{https://www.microsoft.com/en-us/download/details.aspx?id=4865}\\
	\includegraphics[width=1\linewidth]{"pictures/2.4 Network Monitor"}\\
	Die damit empfangenen Pakete können als \texttt{.cap} exportiert werden.\\

	\begin{mynote}
		Eine .cap-Datei (Netzwerk-capture-Datei) und die unterschiedlichen .pml-Dateien (Process Monitor) sind wiederum im .zip-Archiv angehängt.
	\end{mynote}

	\pagebreak
	
	\noindent Folgende Tabelle zeigt oberflächlich die Netzwerk-kommunikation der "a"-Datei (siehe network.csv für mehr Details):\\
	\begin{table}[htbp]
		\centering
		\small
		\begin{tabular}{|m{1cm}|m{1.5cm}|m{3cm}|m{3cm}|m{1.4cm}|}
			\hline
			\textbf{Frame Number} & \textbf{Process Name} &          \textbf{Source} &    \textbf{Destination} & \textbf{Protocol Name} \\ \hline
			15 &        a.exe & 192.168.232.129 & web11.fh-ooe.at &           TCP \\ \hline
			16 &        a.exe & web11.fh-ooe.at & 192.168.232.129 &           TCP \\ \hline
			17 &        a.exe & 192.168.232.129 & web11.fh-ooe.at &           TCP \\ \hline
			18 &        a.exe & 192.168.232.129 & web11.fh-ooe.at &           TLS \\ \hline
			19 &        a.exe & web11.fh-ooe.at & 192.168.232.129 &           TCP \\ \hline
			20 &        a.exe & web11.fh-ooe.at & 192.168.232.129 &           TLS \\ \hline
			21 &        a.exe & web11.fh-ooe.at & 192.168.232.129 &           TCP \\ \hline
			22 &        a.exe & web11.fh-ooe.at & 192.168.232.129 &           TCP \\ \hline
			23 &        a.exe & 192.168.232.129 & web11.fh-ooe.at &           TCP \\ \hline
			24 &        a.exe & web11.fh-ooe.at & 192.168.232.129 &           TLS \\ \hline
			25 &        a.exe & web11.fh-ooe.at & 192.168.232.129 &           TCP \\ \hline
			26 &        a.exe & web11.fh-ooe.at & 192.168.232.129 &           TCP \\ \hline
			27 &        a.exe & 192.168.232.129 & web11.fh-ooe.at &           TCP \\ \hline
			28 &        a.exe & 192.168.232.129 & web11.fh-ooe.at &           TLS \\ \hline
			29 &        a.exe & web11.fh-ooe.at & 192.168.232.129 &           TCP \\ \hline
			30 &        a.exe & web11.fh-ooe.at & 192.168.232.129 &           TLS \\ \hline
			31 &        a.exe & 192.168.232.129 & web11.fh-ooe.at &           TCP \\ \hline
			32 &        a.exe & 192.168.232.129 & web11.fh-ooe.at &           TLS \\ \hline
			33 &        a.exe & web11.fh-ooe.at & 192.168.232.129 &           TCP \\ \hline
			34 &        a.exe & web11.fh-ooe.at & 192.168.232.129 &           TLS \\ \hline
			35 &        a.exe & web11.fh-ooe.at & 192.168.232.129 &           TLS \\ \hline
			36 &        a.exe & web11.fh-ooe.at & 192.168.232.129 &           TCP \\ \hline
			37 &        a.exe & web11.fh-ooe.at & 192.168.232.129 &           TLS \\ \hline
			38 &        a.exe & web11.fh-ooe.at & 192.168.232.129 &           TCP \\ \hline
			39 &        a.exe & 192.168.232.129 & web11.fh-ooe.at &           TCP \\ \hline
			40 &        a.exe & web11.fh-ooe.at & 192.168.232.129 &           TLS \\ \hline
			41 &        a.exe & 192.168.232.129 & web11.fh-ooe.at &           TCP \\ \hline
			42 &        a.exe & web11.fh-ooe.at & 192.168.232.129 &           TLS \\ \hline
			43 &        a.exe & web11.fh-ooe.at & 192.168.232.129 &           TCP \\ \hline
			44 &        a.exe & web11.fh-ooe.at & 192.168.232.129 &           TCP \\ \hline
			45 &        a.exe & 192.168.232.129 & web11.fh-ooe.at &           TCP \\ \hline
			46 &        a.exe & web11.fh-ooe.at & 192.168.232.129 &           TLS \\ \hline
			47 &        a.exe & web11.fh-ooe.at & 192.168.232.129 &           TCP \\ \hline
			48 &        a.exe & 192.168.232.129 & web11.fh-ooe.at &           TCP \\ \hline
			124 &        a.exe & web11.fh-ooe.at & 192.168.232.129 &           TLS \\ \hline
			125 &        a.exe & 192.168.232.129 & web11.fh-ooe.at &           TCP \\ \hline
			126 &        a.exe & web11.fh-ooe.at & 192.168.232.129 &           TCP \\ \hline
			127 &        a.exe & 192.168.232.129 & web11.fh-ooe.at &           TCP \\ \hline
			131 &        a.exe & 192.168.232.129 & web11.fh-ooe.at &           TCP \\ \hline
		\end{tabular}
		\caption{a-File Netzwerkkommunikation}
		\label{table:a-network}
	\end{table}
	

	
	
	\label{LastPage}
	
\end{document}